\section{はじめに}\label{ux306fux3058ux3081ux306b}

あらゆるモノ・コトがプログラムで記述されるような未来が近づいている
プログラムで処理できることは増え続けている
実世界においても、それは言えること
実世界志向インタフェースなどの研究は多くの成果を出している
Hueなどの商品によって、実際に誰でもプログラムできるようになってきている
この流れは変わらず、全世界がプログラム可能になっていくと考えられる。

一方で、コンピュータのみでは処理できないようなこともある
人の確認が必要な状況 人の意志の反映が求められる状況
人でなくてはいけないような状況
こういった問い合わせはプログラムに組み込みづらい
結果として、プログラム化・自動実行などが難しくなっている。

コンピュータの方が得意なこともあれば、人のほうが得意なこともある
何かしらの処理を実現する上で、コンピュータと人は相補的に動作していくべきである
今までは出来なかったようなことの実現

上記の状況を踏まえ、本論文では
プログラムから人へ、処理命令を送れる人力処理組み込み環境 Babascript
を提案する。 Babascript環境では、プログラムというフォーマット上において
コンピュータと人、双方への命令を同様のインタフェースで記述可能になる
特殊なプログラミング言語を導入せず、既存の言語上において実現可能

コンピュータはプログラムという手順書を実行する
人も、常日頃から手順書に基づいて行動している ex: マニュアル、レシピ
手順書のフォーマットが異なるだけで、処理内容が記述された手順書を参照していることには変わらない
つまり、手順書のフォーマットは同一にすることも可能である。

\section{Babascript}\label{babascript}

Babascript環境は、以下の2点を基本構成としたプログラミング環境だ。

\begin{itemize}
\itemsep1pt\parskip0pt\parsep0pt
\item
  Babascript
\item
  Babascript Client
\end{itemize}

これに加え、基本構成の機能を補完するプラグイン機構によって拡張していくことが可能だ。

Babascript環境では、プログラムに人を組み込めるようにするために、関数実行によって人に命令を送れる仕組みを持つ。
従来のプログラミング環境においては、プログラムからコンピュータに対して処理を命令し、処理結果を値として得ている。
Babascript環境では、上記のフローと同じように、プログラムから人に対して処理を命令し、処理結果を値として得られる。
関数というインタフェースを通して、人とコンピュータに対する命令をプログラム上においてほぼ同一のものとすることによって、
プログラム上において簡単に人を扱えるようにする。

以下のような流れで利用可能である。

\begin{enumerate}
\def\labelenumi{\arabic{enumi}.}
\item
  人への命令構文を実行する
\item
  命令構文を元にタスクを生成する
\item
  分散処理基盤を通して適切なクライアントへタスクを配信する
\item
\item
\item
  7.
\end{enumerate}

\subsection{Script}\label{script}

Babascript は、人オブジェクトをプログラム上で扱えるようにしたDSLだ。
人オブジェクトにおいて定義されていない全てのメソッドが人への命令として解釈され、タスクが生成される。
また、命令に対して人からの返り値を得ると、実行メソッドの引数で指定したコールバック関数を実行する。

\begin{verbatim}

    baba = new Baba.Script "baba"
    baba.ほげふが (task) ->
      console.log 'hoge'
\end{verbatim}

人オブジェクトは宣言時にIDを指定することによって、どのクライアントに対して処理を配信するかを決定する。
指定されたIDを監視するクライアントにのみ、タスク情報が配信される。

\subsubsection{オプション情報の付加}\label{ux30aaux30d7ux30b7ux30e7ux30f3ux60c5ux5831ux306eux4ed8ux52a0}

\subsection{Client}\label{client}

Babascript
Clientは、Babascriptからの命令受信と処理結果の送信機能を実現するクライアントライブラリだ。

\subsection{Plugin}\label{plugin}

Babascript
Pluginは、Babascript及びBabascriptClientに組み込むことのできるプラグイン機構だ。
プラグインとして読み込んだ
各種プラグインは、Babascript及びBabascriptClientが発行するイベントとデータを受け取ることができる。
現在対応しているイベントは、以下の通りだ。

\begin{itemize}
\itemsep1pt\parskip0pt\parsep0pt
\item
  init(プラグイン組み込み時)
\item
  connect(ネットワーク接続時)
\item
  send(タスクの送信時)
\item
  receive(タスクの受信時)
\end{itemize}

具体的には、以下のようなプラグインの実装が挙げられる。

\begin{itemize}
\itemsep1pt\parskip0pt\parsep0pt
\item
  ログコレクター
\item
  データ同期
\item
  ユーザ管理
\end{itemize}

\subsection{通信と分散処理}\label{ux901aux4fe1ux3068ux5206ux6563ux51e6ux7406}

scriptとclientのタスク送受信と分散配信のために、node-linda\cite{linda}を利用した。

\subsection{サンプルプログラム}\label{ux30b5ux30f3ux30d7ux30ebux30d7ux30edux30b0ux30e9ux30e0}

\section{実装}\label{ux5b9fux88c5}

\section{応用例}\label{ux5fdcux7528ux4f8b}

以下のような応用が考えられる。

\begin{itemize}
  \item 人の行動をプログラミングする
  \item 実世界をテストする
  \item あれ
\end{itemize}

\subsection{人の行動をプログラミングする}\label{ux4ebaux306eux884cux52d5ux3092ux30d7ux30edux30b0ux30e9ux30dfux30f3ux30b0ux3059ux308b}

\subsection{実世界をテストする}\label{ux5b9fux4e16ux754cux3092ux30c6ux30b9ux30c8ux3059ux308b}

\section{議論}\label{ux8b70ux8ad6}

\subsection{タスク実行の遅延と実行保障性}\label{ux30bfux30b9ux30afux5b9fux884cux306eux9045ux5ef6ux3068ux5b9fux884cux4fddux969cux6027}

Babascriptによってタスク実行を依頼しても、人がすぐにタスクを実行し値を返すことを完全に保証することはできない。
タスク受信端末を見ていない、受信しても実行できないといった状況の場合、すぐに値を返すことはできない。
こういった際、Babascriptによる処理がボトルネックとなる可能性がある。

また、労働関係にあるなど、タスク実行に強制力がある場合は、タスク実行が確実に行われると考えられるが、強制力がない場合はそもそもタスク受信を無視するといったことも考えられる。
タスク実行に強制力がない場合は、金銭などのインセンティブを与えるといった手段によって、実行保障性を確保するといったことが考えられる。

\subsection{命令内容の粒度}\label{ux547dux4ee4ux5185ux5bb9ux306eux7c92ux5ea6}

Babascriptでは、タスクの文面の記述には制限がないため、自由となっている。
この文面は、適切な抽象度の文面に設計しなくてはならない。
抽象度が高すぎる命令は、あいまいな表記となり、タスク実行者にとって理解しづらい文面となり得る。
その結果、想定外の処理が実行され、意図しない結果を招く恐れがある。
抽象度が低すぎる命令は、全体の処理内容にもよるが、プログラム自体が冗長となり得る。
プログラムとタスク実行者の間のやりとりが増え、通信や待機時間などがボトルネックとなる可能性がある。
また、タスク実行者にとっても、やりとりが増えることで負担増になると考えられる。

\subsection{同時の複数命令}\label{ux540cux6642ux306eux8907ux6570ux547dux4ee4}

複数のプログラムから同時に一人のタスク実行者へとタスクが配信される可能性がある。
この際、異なるコンテキストにある命令が交互に配信され、タスク実行に大きな障害をもたらす可能性がある。
例えば、料理プログラムと掃除プログラムが同時に実行された場合、鍋で煮ている途中で「洗剤を投入しろ」などといった命令が配信されることが考えられる。

この問題は、全てのBabascriptプログラム中において、一人のタスク実行者は一つのプログラムからのみ、連続してタスクを受信できるような仕組みを用意することによって、解決可能であると考えられる。
また、応用アプリケーションでの実装になるが、コンテキストを明示し、どの処理系におけるタスクなのかをタスク実行者に示すといった手段によっても解決可能である。

\section{関連研究}\label{ux95a2ux9023ux7814ux7a76}

計算機では処理が難しいようなタスクを解決するために、人を計算資源として利用する手法はヒューマンコンピュテーション\cite{HumanComputation}と呼ばれ、様々な研究が行われている。
インターネットを介して不特定多数の群衆にタスクを実行させるクラウドソーシングと組み合わせた研究事例も多く存在する。
クラウドソーシングのプラットフォームとしては、Amazon Mechanical
Turk\cite{mechanicalturk}が存在する。 Barowy
らは、CrowdProgrammingという概念を提唱し、プログラミング言語内においてクラウドソーシングによる計算とコンピュータによる計算の統合を実現した\cite{automan}。
Franklin
らは、機械だけでは答えられないようなDBへのクエリに対する応答を、クラウドソーシングを用いることで返答可能にするCrowdDBを提案している\cite{crowddb}。
Morishima
らは、人をデータソースとしてプログラムの中で利用する手法を提案している\cite{cylog}。
jabberwocky crowdforge
これらの研究では、人を計算資源やデータソースとしてシステムに組み込むことを狙っている。
本研究では、計算資源やデータソースに限らず、実世界への干渉等も対象としており、本研究はより汎用的な枠組みとなっている。
また、いずれもクラウドソーシングの利用を前提としているが、本研究はクラウドソーシングを対象としたものではない。

ユビキタスコンピューティングの研究分野においては、Human as
Sensorという概念も提唱されている。
PRISMは、スマートフォンを利用したセンシングプラットフォームだ\cite{prism}。
Liuらは、ソーシャルメディア上の人をセンサーとして扱ったQ\&AサービスMoboQを提案し、その検証を行った。
Human as
Sensorに類する研究では、人をセンサーとして扱うことを対象としているが、本研究ではセンサーのみを対象としていない。

Chengらは、人をモーションプラットフォームにおけるモーターやメカニカル機構の代替として利用したHaptic
Turkを提案している\cite{hapticturk}。 Haptic
Turkはゲームでの利用に特化したものだ。
本研究は、使用用途を限らない汎用的な仕組みとなっている。

加藤らは、人とロボット間でのタスク共有システム
Sharedoを提案した\cite{sharedo}。
人とロボットのタスク実行における協調は、本研究の主眼である「何かの処理を実現するとき、人とコンピュータは相補的に動作できるべき」という考えと大きく類似している。

\section{おわりに}\label{ux304aux308fux308aux306b}

本論文では、人への命令構文をプログラムに付与可能なプログラミング環境Babascriptを提案した。
Babascript環境においては、プログラム上において人は、コンピュータと同じ処理ノードとして存在し、関数実行によって処理内容を受け取り、実行・値を返す存在になれる。
これによって、プログラマブルになりつつある世界において人自身もその一部になることが可能となる。

また、Babascript環境によって実現する応用例を示すことによって、人をプログラマブルにした際のメリットを示した。

今後は、議論で述べたBabascriptの問題点などを改善していく。
